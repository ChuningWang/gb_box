\documentclass{article}

\usepackage{graphicx}
\usepackage{epstopdf}
\usepackage{natbib}
\usepackage[margin=1.4in]{geometry}
\usepackage{amsmath}

% \usepackage{apalike}
% \usepackage{captcont}
% \usepackage{fancyref}

\title{An Inverse-Forward Model for Salt Budgets in Glacier Bay, Alaska}
\author{Chuning Wang}

\begin{document}

\maketitle

\section{Introduction}
Box model, both forward and inverse, is a simple but efficient way to establish a first order understanding of the oceanic system \citep{guan2008stommel, stommel1961thermohaline}.

\section{Model Setup}
\label{sec:method}
\subsection{Governing Equations}
The circulation of a Glacier Bay is regulated by several external forcing and constrains: freshwater input, vertical density distribution, wind, tide, and topographic constrains. Topographic constrains include (1) a relatively shallow sill towards the fjord entrance and (2) a relatively narrow channel compared to the length of the fjord. The first constrain conserves the total volume of the fjord;


Volume conservation:
\begin{equation}
  0=F_i+\sum Q_{in}+\sum Q_{out}+\sum W_{in}+\sum W_{out}
  \label{eq:vconserv}
\end{equation}

Salt budgets:
\begin{equation}
  \frac{\rm{d}}{\rm{d}t}S=\sum Q_{in}S_{in}+\sum Q_{out}S_{out}+\sum W_{in}S_{in}+\sum W_{out}S_{out}
  \label{eq:sbudget}
\end{equation}

%----------------------------------------------------------------------------------------
%	BIBLIOGRAPHY
%----------------------------------------------------------------------------------------

\bibliographystyle{apalike}
\bibliography{gb_box}

\end{document}